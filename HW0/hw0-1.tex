\documentclass[12pt, fullpage,letterpaper]{article}

\usepackage[margin=1in]{geometry}
\usepackage{url}
\usepackage{amsmath}
\usepackage{amssymb}
\usepackage{xspace}
\usepackage{graphicx}
\usepackage{xcolor}
%\usepackage{color}

\newcommand{\semester}{Spring 2022}
\newcommand{\assignmentId}{0}
\newcommand{\releaseDate}{10 Jan, 2022}
\newcommand{\dueDate}{11:59pm, 21 Jan, 2022}

\newtheorem{solution}{Solution}

\newcommand{\bx}{{\bf x}}
\newcommand{\bw}{{\bf w}}

\newenvironment{changemargin}[2]{\begin{list}{}{%
\setlength{\topsep}{0pt}%
\setlength{\leftmargin}{0pt}%
\setlength{\rightmargin}{0pt}%
\setlength{\listparindent}{\parindent}%
\setlength{\itemindent}{\parindent}%
\setlength{\parsep}{0pt plus 1pt}%
\addtolength{\leftmargin}{#1}%
\addtolength{\rightmargin}{#2}%
}\item }{\end{list}}

\def\red{\color{red}}
\def\blue{\color{blue}}
\def\blackblue{\color{black!40!blue}}

\title{CS 6190: Probabilistic Machine Learning \semester}
\author{Solutions to Homework \assignmentId\\
by: \textcolor{red}{Meysam Alishahi} \\ 
\textcolor{red}{(UNID: u01323606)}
}
\date{Handed out: \releaseDate\\
  Due: \dueDate}

\begin{document}
\maketitle

% Math commands by Thomas Minka
\newcommand{\var}{{\rm var}}
\newcommand{\Tr}{^{\rm T}}
\newcommand{\vtrans}[2]{{#1}^{(#2)}}
\newcommand{\kron}{\otimes}
\newcommand{\schur}[2]{({#1} | {#2})}
\newcommand{\schurdet}[2]{\left| ({#1} | {#2}) \right|}
\newcommand{\had}{\circ}
\newcommand{\diag}{{\rm diag}}
\newcommand{\invdiag}{\diag^{-1}}
\newcommand{\rank}{{\rm rank}}
% careful: ``null'' is already a latex command
\newcommand{\nullsp}{{\rm null}}
\newcommand{\tr}{{\rm tr}}
\renewcommand{\vec}{{\rm vec}}
\newcommand{\vech}{{\rm vech}}
\renewcommand{\det}[1]{\left| #1 \right|}
\newcommand{\pdet}[1]{\left| #1 \right|_{+}}
\newcommand{\pinv}[1]{#1^{+}}
\newcommand{\erf}{{\rm erf}}
\newcommand{\hypergeom}[2]{{}_{#1}F_{#2}}

% boldface characters
\renewcommand{\a}{{\bf a}}
\renewcommand{\b}{{\bf b}}
\renewcommand{\c}{{\bf c}}
\renewcommand{\d}{{\rm d}}  % for derivatives
\newcommand{\e}{{\bf e}}
\newcommand{\f}{{\bf f}}
\newcommand{\g}{{\bf g}}
\newcommand{\h}{{\bf h}}
%\newcommand{\k}{{\bf k}}
% in Latex2e this must be renewcommand
\renewcommand{\k}{{\bf k}}
\newcommand{\m}{{\bf m}}
\newcommand{\mb}{{\bf m}}
\newcommand{\n}{{\bf n}}
\renewcommand{\o}{{\bf o}}
\newcommand{\p}{{\bf p}}
\newcommand{\q}{{\bf q}}
\renewcommand{\r}{{\bf r}}
\newcommand{\s}{{\bf s}}
\renewcommand{\t}{{\bf t}}
\renewcommand{\u}{{\bf u}}
\renewcommand{\v}{{\bf v}}
\newcommand{\w}{{\bf w}}
\newcommand{\x}{{\bf x}}
\newcommand{\y}{{\bf y}}
\newcommand{\z}{{\bf z}}
%s\newcommand{\l}{\boldsymbol{l}}
\newcommand{\A}{{\bf A}}
\newcommand{\B}{{\bf B}}
\newcommand{\C}{{\bf C}}
\newcommand{\D}{{\bf D}}
\newcommand{\E}{{\bf E}}
\newcommand{\F}{{\bf F}}
\newcommand{\G}{{\bf G}}
\renewcommand{\H}{{\bf H}}
\newcommand{\I}{{\bf I}}
\newcommand{\J}{{\bf J}}
\newcommand{\K}{{\bf K}}
\renewcommand{\L}{{\bf L}}
\newcommand{\M}{{\bf M}}
\newcommand{\N}{\mathcal{N}}  % for normal density
\newcommand{\Dcal}{\mathcal{D}}  % for normal density

%\newcommand{\N}{{\bf N}}
\renewcommand{\O}{{\bf O}}
\renewcommand{\P}{{\bf P}}
\newcommand{\Q}{{\bf Q}}
\newcommand{\R}{{\bf R}}
\renewcommand{\S}{{\bf S}}
\newcommand{\T}{{\bf T}}
\newcommand{\U}{{\bf U}}
\newcommand{\V}{{\bf V}}
\newcommand{\W}{{\bf W}}
\newcommand{\X}{{\bf X}}
\newcommand{\Y}{{\bf Y}}
\newcommand{\Z}{{\bf Z}}

% this is for latex 2.09
% unfortunately, the result is slanted - use Latex2e instead
%\newcommand{\bfLambda}{\mbox{\boldmath$\Lambda$}}
% this is for Latex2e
\newcommand{\bfLambda}{\boldsymbol{\Lambda}}

% Yuan Qi's boldsymbol
\newcommand{\bsigma}{\boldsymbol{\sigma}}
\newcommand{\balpha}{\boldsymbol{\alpha}}
\newcommand{\bpsi}{\boldsymbol{\psi}}
\newcommand{\bphi}{\boldsymbol{\phi}}
\newcommand{\boldeta}{\boldsymbol{\eta}}
\newcommand{\Beta}{\boldsymbol{\eta}}
\newcommand{\btau}{\boldsymbol{\tau}}
\newcommand{\bvarphi}{\boldsymbol{\varphi}}
\newcommand{\bzeta}{\boldsymbol{\zeta}}

\newcommand{\blambda}{\boldsymbol{\lambda}}
\newcommand{\bLambda}{\mathbf{\Lambda}}
\newcommand{\bOmega}{\mathbf{\Omega}}
\newcommand{\bomega}{\mathbf{\omega}}
\newcommand{\bPi}{\mathbf{\Pi}}

\newcommand{\btheta}{\boldsymbol{\theta}}
\newcommand{\bpi}{\boldsymbol{\pi}}
\newcommand{\bxi}{\boldsymbol{\xi}}
\newcommand{\bSigma}{\boldsymbol{\Sigma}}

\newcommand{\bgamma}{\boldsymbol{\gamma}}
\newcommand{\bGamma}{\mathbf{\Gamma}}

\newcommand{\bmu}{\boldsymbol{\mu}}
\newcommand{\1}{{\bf 1}}
\newcommand{\0}{{\bf 0}}

% \newcommand{\comment}[1]{}

\newcommand{\bs}{\backslash}
\newcommand{\ben}{\begin{enumerate}}
\newcommand{\een}{\end{enumerate}}

 \newcommand{\notS}{{\backslash S}}
 \newcommand{\nots}{{\backslash s}}
 \newcommand{\noti}{{\backslash i}}
 \newcommand{\notj}{{\backslash j}}
 \newcommand{\nott}{\backslash t}
 \newcommand{\notone}{{\backslash 1}}
 \newcommand{\nottp}{\backslash t+1}
% \newcommand{\notz}{\backslash z}

\newcommand{\notk}{{^{\backslash k}}}
%\newcommand{\noti}{{^{\backslash i}}}
\newcommand{\notij}{{^{\backslash i,j}}}
\newcommand{\notg}{{^{\backslash g}}}
\newcommand{\wnoti}{{_{\w}^{\backslash i}}}
\newcommand{\wnotg}{{_{\w}^{\backslash g}}}
\newcommand{\vnotij}{{_{\v}^{\backslash i,j}}}
\newcommand{\vnotg}{{_{\v}^{\backslash g}}}
\newcommand{\half}{\frac{1}{2}}
\newcommand{\msgb}{m_{t \leftarrow t+1}}
\newcommand{\msgf}{m_{t \rightarrow t+1}}
\newcommand{\msgfp}{m_{t-1 \rightarrow t}}

\newcommand{\proj}[1]{{\rm proj}\negmedspace\left[#1\right]}
\newcommand{\argmin}{\operatornamewithlimits{argmin}}
\newcommand{\argmax}{\operatornamewithlimits{argmax}}

\newcommand{\dif}{\mathrm{d}}
\newcommand{\abs}[1]{\lvert#1\rvert}
\newcommand{\norm}[1]{\lVert#1\rVert}

%miscellaneous symbols
\newcommand{\ie}{{{i.e.,}}\xspace}
\newcommand{\eg}{{{\em e.g.,}}\xspace}
\newcommand{\EE}{\mathbb{E}}
\newcommand{\VV}{\mathbb{V}}
\newcommand{\sbr}[1]{\left[#1\right]}
\newcommand{\rbr}[1]{\left(#1\right)}
\newcommand{\cmt}[1]{}



\footnotesize
	\begin{itemize}
		\item You are welcome to talk to other members of the class about
		the homework. I am more concerned that you understand the
		underlying concepts. However, you should write down your own
		solution. Please keep the class collaboration policy in mind.
		
		\item Feel free discuss the homework with the instructor or the TAs.
		
		\item Your written solutions should be brief and clear. You need to
		show your work, not just the final answer, but you do \emph{not}
		need to write it in gory detail. Your assignment should be {\bf no
			more than 10 pages}. Every extra page will cost a point.
		
		\item Handwritten solutions will not be accepted.
		
		\item The homework is due by \textbf{midnight of the due date}. Please submit
		the homework on Canvas.
	\end{itemize}



\section*{Warm up[100 points + 5 bonus]}	
\label{sec:q1}

%independency, conditional distribution, expectation, variance, basic properties
%gradient calcualtion, logistic function, second derivatives
%
\begin{enumerate}
%\item~[5 points] We use sets to represent events. For example, toss a fair coin $5$ times, and the event can be represented by the set of ``Heads" or ``Tails" after each tossing. Let a specific event $A$ be ``at least one head after tossing $5$ times". Calculate the probability that event $A$ happens, i.e., $p(A)$.


\item~[2 points] Given two events $A$ and $B$, prove that 
\begin{align}
&p(A \cup B) \le p(A) + p(B) \nonumber \\
&p(A \cap B) \le p(A), p(A \cap B) \le p(B) \nonumber
\end{align}
When does the equality hold?

{\blackblue In view of the third probability Axiom, known as the assumption of $\sigma$-additivity, we have 
\begin{align}\label{q1}
p(A\cup B) = P(A\setminus B) +  P(B) \qquad\qquad \mbox{and}\qquad\qquad p(A) = p(A\setminus B) + p(A\cap B)
\end{align}
since  $(A\setminus B)\cap B = \varnothing$ and $(A\setminus B)\cap B=\varnothing$.  
Since $p(B)$ is non-negative, these equalities imply 
\begin{align*}
p(A\cup B) 
& =  p(A\setminus B) +  p(B)\\
& = p(A) - p(A\cap B)  + p(B)\\
& \leq  p(A) + p(B).
\end{align*}
To have the equality,  we need $p(A) - p(A\cap B)  + p(B) = p(A) + p(B)$ which concludes $p(A\cap B) = 0$. 
In particular, when $A\cap B = \varnothing$ we have the equlity. 


In view of the second equality in Equality~(\ref{q1}), 
\begin{align*}
p(A\cap B) &= p(A) - p(A\setminus B)\\
		 & \leq p(A).
\end{align*}
Again, to have the equality, we should have  $p(A) - p(A\setminus B) = p(A)$ or equivalently, $p(A\setminus B)= 0$. In particular, when $A\subseteq B$, the equality holds.
With a similar approach, we can prove $p(A\cap B)\leq p(B)$ and for which the equality holds if and only if $p(B\setminus A)= 0$. In particular, when $B\subseteq A$, we have the equality.
}
\item~[2 points] Let $\{A_1, \ldots, A_n\}$ be a collection of events. Show that
\[
p(\cup_{i=1}^n A_i) \le \sum_{i=1}^n p(A_i).
\]
When does the equality hold? (Hint: induction)\\
{\blackblue We are going to prove the following statement:

\begin{changemargin}{1cm}{1cm}
\noindent{\it If $\{A_1, \ldots, A_n\}$ are a collection of events, then
$p(\cup_{i=1}^n A_i) \le \sum_{i=1}^n p(A_i)$
and equality holds if and only if $p(A_i\cap A_j)=0$ for each $i\neq j\in\{1,\ldots,n\}$.
In particular, if $A_1, \ldots, A_n$ are pairwise disjoint, i.e., $A_i\cap A_j=\varnothing$ for each $i\neq j\in\{1,\ldots,n\}$, then we have the equality. }
\end{changemargin} 
{\noindent \bf Proof.}
The induction base ($n=2$) is holding due to Exercise  1. Assume that the statement is true for $n=k\geq 2$ and we want to prove it for $n = k+1$. 
Set $A = \cup_{i=1}^k A_i$ and $B=A_{k+1}$. By induction base and hypothesis, we have 
\begin{align*}
p(\cup_{i=1}^n A_i) 
& =  p(A\cup B)\\
&\leq p(A) + p(B)\\
& = p(\cup_{i=1}^k A_i) + p(A_{k+1})\\
&\leq \sum_{i=1}^k p(A_i) +  p(A_{k+1}).
\end{align*}
To have the equality, we should have 
$$ p(A\cup B) = p(A) + p(B)\qquad\qquad \mbox{and} \qquad\qquad p(\cup_{i=1}^k A_i) = \sum_{i=1}^k p(A_i).$$
Again using induction base, we conclude 
$p((\cup_{i=1}^k A_i)\cap A_{k+1})= 0,$ which implies $p(A_i\cap A_{k+1})=0$ for each $i\in\{1,\cdots,k\}$. 
Also, from the induction hypothesis, since $p(\cup_{i=1}^k A_i) = \sum_{i=1}^k p(A_i),$
we got 
$$p(A_i\cap A_j)=0 \quad \mbox{for each $i\neq j\in \{1,\dots,k\}.$}$$
Overall,  we proved that 
if $p(\cup_{i=1}^{k+1} A_i) = \sum_{i=1}^{k+1} p(A_i),$
then
$$p(A_i\cap A_j)=0 \quad \mbox{for each $i\neq j\in \{1,\dots,k+1\}$}.$$
}

%\item~[5 points] Given three events $A$, $B$ and $C$, show that
%\[
%p(A\cap B\cap C) = p(A|B\cap C)p(B|C)p(C)
%\]
\item~[14 points]  We use $\EE(\cdot)$ and $\VV(\cdot)$ to denote a random variable's mean (or expectation) and variance, respectively. Given two discrete random variables $X$ and $Y$, where $X \in \{0, 1\}$ and $Y \in \{0,1\}$. The joint probability $p(X,Y)$ is given in as follows:
\begin{table}[h]
        \centering
        \begin{tabular}{ccc}
        \hline\hline
         & $Y=0$ & $Y=1$ \\ \hline
         $X=0$ & $3/10$ & $1/10$ \\ \hline
         $X=1$  & $2/10$ & $4/10$ \\ \hline\hline
        \end{tabular}
        %\caption{Training data for the alien invasion problem.}\label{tb-alien-train}
        \end{table}
	
        \begin{enumerate}
            \item~[10 points] Calculate the following distributions and statistics. 
            \begin{enumerate}
            \item the the marginal distributions $p(X)$ and $p(Y)$
            {\blackblue             
            
            \noindent Marginal distributions $p(X)$:
            $$p(X=0) = p(X=0, Y=0) + p(X=0, Y=1) = 3/10+1/10 = 4/10$$
            $$p(X=1) = p(X=1, Y=0) + p(X=1, Y=1) = 2/10+4/10 = 6/10$$
            
            \noindent Marginal distributions $p(Y)$:
            $$p(Y=0) = p(X=0, Y=0) + p(X=1, Y=0) = 3/10+2/10 = 5/10$$
            $$p(Y=1) = p(X=0, Y=1) + p(X=0, Y=1) = 1/10+4/10 = 5/10$$
            }
            \item the conditional distributions $p(X|Y)$ and $p(Y|X)$
             {\blackblue  
             \noindent Conditional distributions $p(X|Y)$:
             $$p(X=0|Y=0) =3/5 \qquad \qquad \mbox{and}\qquad\qquad p(X=1|Y=0) = 2/5$$
             $$p(X=0|Y=1) =1/5 \qquad \qquad \mbox{and}\qquad\qquad p(X=1|Y=1) = 4/5$$
             $$p(Y=0|X=0) =3/4 \qquad \qquad \mbox{and}\qquad\qquad p(Y=1|X=0) =1/4 $$
             $$p(Y=0|X=1) =1/3 \qquad \qquad \mbox{and}\qquad\qquad p(Y=1|X=1) = 2/3$$
            }
            \item $\EE(X)$, $\EE(Y)$, $\VV(X)$, $\VV(Y)$\\
            {\blackblue  
             \noindent 
             $$\EE(X) = 0.6  \qquad \qquad \mbox{and}\qquad\qquad  \EE(Y) = 0.5$$
             $$\VV(X) = \EE(X^2) - \EE(X)^2 = 0.6 - 0.36 = 0.24$$ 
             $$\VV(X) = \EE(Y^2) - \EE(Y)^2 = 0.5 - 0.25 = 0.25$$
             }
            \item  $\EE(Y|X=0)$, $\EE(Y|X=1)$,  $\VV(Y|X=0)$, $\VV(Y|X=1)$ \\
            {\blackblue  
             \noindent 
             $$\EE(Y|X=0) = 1/4 \qquad \qquad \mbox{and}\qquad\qquad \EE(Y|X=1) = 2/3$$
             $$\VV(Y|X=0) = \EE(Y^2|X=0) - \EE(Y|X=0)^2 = 1/4 - 1/16 = 3/16$$
             $$\VV(Y|X=1) = \EE(Y^2|X=1) - \EE(Y|X=1)^2 = 2/3 - 4/9 = 2/9$$
            }
            \item  the covariance between $X$ and $Y$\\
            {\blackblue  
             \noindent 
             $$cov(X,Y) = \EE(XY) - \EE(X)\EE(Y) = 0.4 - 0.6 \times 0.5 = 0.37$$ 
             }
            \end{enumerate}
            \item~[2 points] Are $X$ and $Y$ independent? Why?
            {\blackblue  
             \noindent No, since $$0.3 = p(X=0,Y=0)  \neq p(X=0)p(Y=0) = 0.4\times 0.5 = 0.2$$}
      
            \item~[2 points] When $X$ is not assigned a specific value, are $\EE(Y|X)$ and $\VV(Y|X)$ still constant? Why?
            {\blackblue  
             \noindent No, since $X$ and $Y$ are not independent, $\EE(Y|X)$ and $\VV(Y|X)$ are both functions of $X$ and therefore, they are two random variables depending on $X$. 
            }
            
        \end{enumerate}
\item~[9 points] Assume a random variable $X$ follows a standard normal distribution, \ie $X \sim \N(X|0, 1)$. Let $Y = e^{-X^2}$. Calculate the mean and variance of $Y$.
\begin{enumerate}
	\item $\EE(Y)$\\
	{\blackblue  
	Note that if $Z \sim \N(Z|\mu, \sigma)$, then the probability density function of $Z$ is 
	$f_Z(z) = {1\over \sigma\sqrt{2\pi}}e^{-{(z-\mu)^2\over 2\sigma^2}}.$
	\begin{align*}
	\EE(Y) 
	& = \int_{-\infty}^{+\infty}e^{-X^2}{1\over \sqrt{2\pi}}e^{-{X^2\over 2}}dx\\
	& = {1\over \sqrt{3}}\int_{-\infty}^{+\infty}{1\over {1\over \sqrt{3}}\sqrt{2\pi}}e^{-{X^2\over 2/3}}dx\\
	& = {1\over \sqrt{3}}\times 1 = {1\over \sqrt{3}}.
	\end{align*}
	Note that $g_X(x) = {1\over {1\over \sqrt{3}}\sqrt{2\pi}}e^{-{x^2\over 2/3}}$ is the probability density function of  $\N(X|0, 1/\sqrt{3})$ and thus $\int_{-\infty}^{+\infty}{1\over {1\over \sqrt{3}}\sqrt{2\pi}}e^{-{X^2\over 2/3}}dx =1 $.
	}
	
	\item $\VV(Y)$\\
	{\blackblue To compute $\VV(Y)= \EE(Y^2) - \EE(Y)^2$, we need to know $\EE(Y^2)$. Lets compute it,
	\begin{align*}
	\EE(Y^2) 
	& = \int_{-\infty}^{+\infty}e^{-2X^2}{1\over \sqrt{2\pi}}e^{-{X^2\over 2}}dx\\
	& = {1\over \sqrt{5}}\int_{-\infty}^{+\infty}{1\over {1\over \sqrt{5}}\sqrt{2\pi}}e^{-{X^2\over 2/5}}dx\\
	& = {1\over \sqrt{5}}\times 1 = {1\over \sqrt{5}}.
	\end{align*}
	Therefore, 
	$$\VV(Y) = \EE(Y^2) - \EE(Y)^2 = {1\over \sqrt{5}} -({1\over \sqrt{3}})^2 = {1\over \sqrt{5}} - {1\over 3}$$

	}
	\item $\text{cov}(X, Y)$\\
	{\blackblue\noindent We know that $\text{cov}(X, Y)=\EE(XY) - \EE(X)\EE(Y)$.
	As we know  $\EE(X) = 0$ and $\EE(Y)= {1\over \sqrt{3}}$, to compute \text{cov}(X, Y), it suffices to know the value of $\EE(XY)$.
	\begin{align*}
	\EE(XY) 
	& = \int_{-\infty}^{+\infty}Xe^{-X^2}{1\over \sqrt{2\pi}}e^{-{X^2\over 2}}dx\\
	& = {1\over \sqrt{3}}\int_{-\infty}^{+\infty}X{1\over {1\over \sqrt{3}}\sqrt{2\pi}}e^{-{X^2\over 2/3}}dx\\
	& = {1\over \sqrt{3}}\EE(X)\qquad\mbox{where $X\sim \N(X|0, 1/\sqrt{3})$}\\
	& = 0.
	\end{align*}
	Therefore, 
	$\text{cov}(X, Y)=\EE(XY) - \EE(X)\EE(Y) = 0 - 0\times  {1\over \sqrt{3}} = 0.$
	}
\end{enumerate}

\item~[8 points] Derive the probability density functions of the following transformed random variables. 
\begin{enumerate}
	\item $X \sim \N(X|0, 1)$ and $Y = X^3$.\\
	{\blackblue\noindent 
	We remind that if $X \sim f(X)$, then by changing the variable $X = g(Y)$ where $g$ is monotonic function,  $Y\sim g'(Y) f(g(Y))$.
	Therefore, since here $g(Y) = Y^{1/3}$ and $X\sim \frac{1}{\sqrt{2\pi}}e^{-\frac{X^2}{2}}$, we have 
	\begin{align*}
	Y & \sim \frac{1}{3}Y^{-2/3}\frac{1}{\sqrt{2\pi}}e^{-\frac{Y^{2/3}}{2}} = \frac{1}{3\sqrt{2\pi}\sqrt[3]{Y^2}}e^{-\frac{Y^{2/3}}{2}}.
	\end{align*}
	}
	%\item $\x =[x_1, x_2]^\top \sim \N(\x|\0, [1, -1/2;1/2,1])$ and $y =3 x_1 + 2x_2$. 
	\item $\left[\begin{array}{c}X_1 \\ X_2\end{array}\right] \sim \N\big(\left[\begin{array}{c}X_1 \\ X_2\end{array}\right]|\left[\begin{array}{c}0 \\ 0 \end{array}\right], \left[\begin{array}{cc}1 & -1/2 \\ -1/2 & 1\end{array}\right]\big)$ and $\left[\begin{array}{c}Y_1 \\ Y_2\end{array}\right] = \left[\begin{array}{cc}1 & 1/2\\-1/3&1\end{array}\right]\left[\begin{array}{c}X_1 \\ X_2\end{array}\right]$.\\
	{\blackblue\noindent 
	We remind that if $\x\sim \N(\boldsymbol{\mu},\boldsymbol{\Sigma})$ and $\y = \A\x +\b$, then $\y\sim \N(\A\boldsymbol{\mu} +\b, \A\boldsymbol{\Sigma}\A^\top).$
	Therefore, $\left[\begin{array}{c}Y_1 \\ Y_2\end{array}\right]$ has a Gaussian distribution with mean
	$$\left[\begin{array}{cc}1 & 1/2\\-1/3&1\end{array}\right]\left[\begin{array}{c}0 \\ 0\end{array}\right] = \left[\begin{array}{c}0 \\ 0\end{array}\right]$$
	and covariance 
	\begin{align*}
	\left[\begin{array}{cc}1 & 1/2\\-1/3&1\end{array}\right] \left[\begin{array}{cc}1 & -1/2 \\ -1/2 & 1\end{array}\right]\left[\begin{array}{cc}1 & 1/2\\-1/3&1\end{array}\right]^\top
	%& = \left[\begin{array}{cc}1 & 1/2\\-1/3&1\end{array}\right] \left[\begin{array}{cc}1 & -1/2 \\ -1/2 & 1\end{array}\right]\left[\begin{array}{cc}1 & -1/3\\1/2&1\end{array}\right]\\
	%& = \left[\begin{array}{cc}1 & 1/2\\-1/3&1\end{array}\right] \left[\begin{array}{cc}3/4 & -5/6 \\ 0 & 7/6\end{array}\right]\\
	& = 
	\left[\begin{array}{cc}3/4 & -1/4\\-1/4&13/9\end{array}\right].
	\end{align*}
	In other words, 
	$$\left[\begin{array}{c}Y_1 \\ Y_2\end{array}\right] \sim \N\left(\left[\begin{array}{c}Y_1 \\ Y_2\end{array}\right]| \left[\begin{array}{c}0 \\ 0\end{array}\right], \left[\begin{array}{cc}3/4 & -1/4\\-1/4&13/9\end{array}\right]\right).$$
	}
\end{enumerate}


\item~[10 points]  Given two random variables $X$ and $Y$, show that 
\begin{enumerate}
\item $\EE(\EE(Y|X)) = \EE(Y)$
{\blackblue\noindent 
\begin{align*}
\EE(\EE(Y|X)) 
& = \int \EE(Y|X=x)f_X(x)dx \\
& = \int \left(\int yf_{Y|X}(y|x)dy\right)f_X(x)dx \\
& = \int \int yf_X(x)f_{Y|X}(y|x)dydx \\
& = \int \int yf_{X,Y}(x,y)dydx \\
& = \int y\left(\int f_{X,Y}(x,y)dx\right)dy \\
& = \int yf_Y(y)dy = \EE(Y) \qquad\qquad \mbox{(we used the fact that $f_Y(y) = \int F_{X,Y}(x,y)dx$)}
\end{align*}}
\item
$\VV(Y) = \EE(\VV(Y|X)) + \VV(\EE(Y|X))$

{\blackblue\noindent 
\begin{align*}
\VV(Y) & = \EE(Y^2) - \EE(Y)^2\\\
& = \EE(\EE(Y^2|X)) - \EE(\EE(Y|X))^2\\
& = \EE\Big(\EE(Y^2|X)-\EE(Y|X)^2+\EE(Y|X)^2\Big) - \EE(\EE(Y|X))^2\\
& = \EE(\EE(Y^2|X))-\EE(Y|X)^2) +\EE(\EE(Y|X)^2) - \EE(\EE(Y|X))^2 \qquad \mbox{linearity of expectation} \\
& =\EE(\VV(Y|X)) + \VV(\EE(Y|X)).
\end{align*}}
\end{enumerate}

%\item~[20 points]  Let us go back to the coin tossing example. Suppose we toss a coin for $n$ times, \textit{independently}. Each toss we have $\frac{1}{2}$ chance to obtain the head. Let us denote the total number of heads by $c(n)$. Derive the following statistics. You don't need to give the numerical values. You only need to provide the formula.
%\begin{enumerate}
%\item $\EE(c(1))$, $\VV(c(1))$
%\item $\EE(c(10))$, $\VV(c(10))$
%\item $\EE(c(n))$, $\VV(c(n))$
%\end{enumerate} 
%What can you conclude from comparing the expectations and variances with different choices of $n$?  

\item~[9 points] Given a logistic function, $f(\x) = 1/(1+\exp(-\a^\top \x))$ ($\x$ is a vector),  
\begin{enumerate}
\item derive $\frac{\d f(\x)}{\d \x}$\\
{\blackblue\noindent 
We know $\frac{\d \sigma}{\d z} = \sigma(z)(1-\sigma(z))$ if $\sigma(z) = 1/(1+\exp(-z))$. 
Set $z= \a^\top \x$ and note that $f(\x) = \sigma(z)$.
%\begin{align*}
%\frac{\d f(\x)}{\d \x} &= \frac{\d z}{\d \x}\frac{\d \sigma}{\d z}\\
%& = \a\sigma(z)(1-\sigma(z))\\
%& = f(\x)(1-f(\x))\a 
%\end{align*}
\begin{align*}
\d f(\x)&= \frac{\d \sigma}{\d z}\d z\\
& = \sigma(z)(1-\sigma(z))\d(\a^\top\x)\\
& = \sigma(z)(1-\sigma(z))\a^\top\d\x \quad\Longrightarrow {\d f(\x)\over \d\x} = \sigma(z)(1-\sigma(z))\a^\top
\end{align*}}
\item derive $\frac{\d^2 f(\x)}{\d \x^2}$, \ie the Hessian matrix\\
{\blackblue\noindent 
Set $t(\x) = f(\x)(1-f(\x))$.
The vector $\a$ is $n\times 1$ and here, we consider $t$ as a $1\times 1$ matrix. 
Clearly, $\frac{\d f(\x)}{\d \x}  = t(\x)\a^\top$.
Therefore, 
\begin{align*}
\d\left(\frac{\d f(\x)}{\d \x}\right)
& = \d \left(\a t(\x)\right)\\
& = \a(\d t(\x))\\
& = \a f(\x)(1-f(\x))(1-2f(\x))\a^\top\d\x \quad\Longrightarrow \frac{\d^2 f(\x)}{\d \x^2}  =f(\x)(1-f(\x))(1-2f(\x)) \a \a^\top
\end{align*}
}
\item show that $-\log f(\x)$ is convex 
Note that $0 \le f(\x) \le 1$.\\
{\blackblue\noindent 
To prove that $-\log f(\x)$ is convex, we use a result  asserting that a function $g(\x)$ is convex if its Hessian matrix $H(\x)$ is semi-positive for each $x$, i.e, 
$\frac{\d^2 g(\x)}{\d \x^2} \succeq 0$. 
\begin{align*}
\frac{\d }{\d \x} (-\log f(\x)) 
& = -\frac{1}{f(\x)}\frac{\d f(\x)}{\d \x} = -(1-f(\x))\a \quad\quad\mbox{Using Part (a)}
\end{align*}
Therefore, 
\begin{align*}
\frac{\d^2 }{\d \x^2} (-\log f(\x)) 
& = \frac{\d }{\d \x} \Big(\frac{\d }{\d \x} (-\log f(\x))\Big)\\
& = \frac{\d }{\d \x} \Big(-(1-f(\x))\a \Big)\\
& = f(\x)(1-f(\x))\a \a^t
\end{align*}
Notice $A_{n\times n}$ is semi-positive if and only if for each vector $\z\in\mathbb{R}^n$, we have $z^\top A \z\geq 0$.
\begin{align*}
\z^\top \Big(f(\x)(1-f(\x))\a \a^\top\Big) \z
& = f(\x)(1-f(\x)) \z^\top\a \a ^\top\z\\
& =  f(\x)(1-f(\x)) \z^\top\a( \z^\top \a )^\top\\
& = f(\x)(1-f(\x))  (\z^\top\a)^2 \geq 0 \quad\quad \mbox{\red We used $f(\x)(1-f(\x))\geq 0$ as well.}
\end{align*}
}
\end{enumerate}
\item ~[10 points] Derive the convex conjugate for the following functions
\begin{enumerate}
\item $f(x) = -\log(x)$\\
{\blackblue\noindent 
By the definition, the convex conjugate of $f(x)$ is 
$f^*(\lambda) = \displaystyle\max_{x\in (0,\infty)} \left(\lambda x + \log x\right)$. 
Note that since, for each arbitrary fixed $\lambda\in \mathbb{R}$,  $\lambda x + \log x$ is a strictly concave function, either it takes its maximum in a unique point (can be found by setting its derivative to zero) or its maximum is $+\infty$. 
When $\lambda \geq 0$, clearly $\displaystyle\max_{x} \left(\lambda x + \log x\right) = \infty$. 
When $\lambda < 0$, if we set the derivative of $\displaystyle\max_{x\in (0,\infty)} \left(\lambda x + \log x\right)$ to zero, then
$$\left(\lambda x + \log x\right)' = \lambda + 1/x = 0\quad\quad \Longrightarrow\quad\quad x = -1/\lambda.$$
Substituting $x = -1/\lambda$, we get 
$\displaystyle\max_{x\in (0,\infty)} \left(\lambda x + \log x\right) = -(1 + \log(\lambda)).$
Therefore, 
$$f^*(\lambda) = \left\{
\begin{array}{ll}
+\infty & \lambda\geq 0\\
-(1 + \log(\lambda)) & \lambda< 0
\end{array}\right.$$
}
\item $f(\x) = \x^\top \A^{-1} \x$ where $\A \succ 0$\\
{\blackblue\noindent 
By the definition, 
$$f^*(\y) = \displaystyle\max_{\x\in\mathbb{R}^n} \left(\y^\top \x - \x^\top\A^{-1}\x\right).$$
Since $\left(\y^\top \x - \x^\top\A^{-1}\x\right)$ is strictly concave, it takes its maximum in a point making its derivative zero. 
$$\frac{\d}{\d \x} \left(\y^\top \x - \x^\top\A^{-1}\x\right) = \y- 2 \A^{-1}\x = 0 \quad \Longrightarrow \x = \frac{1}{2} \A\y.$$
Substituting $\x = \frac{1}{2} \A\y$, we get 
$f^*(\y) =  \frac{1}{4}\y^\top \A\y.$
}
\end{enumerate}

\item~[20 points] Derive the (partial) gradient of the following functions. Note that bold small letters represent vectors, bold captial letters matrices, and non-bold letters just scalars.
\begin{enumerate}
	\item $f(\x) = \x^\top \A \x$, derive $\frac{\partial f}{\partial \x}$
	{\blackblue\noindent  
	\begin{align*}
	\d f(\x) & = \d \left(\x^\top \A \x\right)\\
	& = (\d \x)^\top  \A \x + \x^\top \d(\A \x)\\
	& = (\d \x)^\top  \A \x + \x^\top \A \d \x\\
	& = \x^\top \A^\top \d \x + \x^\top \A \d \x \quad\quad\mbox{Since $(\d \x)^\top  \A \x$ is scaler, it is equal to its transpose.}\\
	& =  (\x^\top \A^\top + \x^\top \A) \d \x \quad\Longrightarrow \quad {\red \frac{\partial f}{\partial \x} =   \x^\top (\A + \A^\top)}
	\end{align*}}
	\item $f(\x)  = \left(\I + \x\x^\top\right)^{-1} \x$, derive $\frac{\partial f}{\partial \x}$\\
	{\blackblue \noindent
	\begin{align*}
	\d f(\x) & =\d \left(\left(\I + \x\x^\top\right)^{-1} \x\right)\\
	& = \d \left(\I + \x\x^\top\right)^{-1} \x + \left(\I + \x\x^\top\right)^{-1} \d\x\\
	& = - \left(\I + \x\x^\top\right)^{-1} \d \left(\I + \x\x^\top\right) \left(\I + \x\x^\top\right)^{-1} \x + \left(\I + \x\x^\top\right)^{-1} \d\x\\
	& = - \left(\I + \x\x^\top\right)^{-1} \left((\d\x)\x^\top +\x(\d \x)^\top\right) \left(\I + \x\x^\top\right)^{-1} \x + \left(\I + \x\x^\top\right)^{-1} \d\x\\
	& =  - \left(\I + \x\x^\top\right)^{-1} (\d\x)\underbrace{\red\x^\top  \left(\I + \x\x^\top\right)^{-1} \x}_\text{Scaler} -
	 \left(\I + \x\x^\top\right)^{-1}  \x\underbrace{{\red(\d \x)^\top \left(\I + \x\x^\top\right)^{-1} \x}}_\text{Scaler}
	 + \left(\I + \x\x^\top\right)^{-1} \d\x\\
	& =  - \underbrace{\red\left(\x^\top  \left(\I + \x\x^\top\right)^{-1} \x\right)}_\text{Scaler}\left(\I + \x\x^\top\right)^{-1} (\d\x) -
	 \left(\I + \x\x^\top\right)^{-1}  \x{\red\x^\top \left(\I + \x\x^\top\right)^{-1} \d \x}
	 + \left(\I + \x\x^\top\right)^{-1} \d\x\\
	& =  \left[-\left(\x^\top  \left(\I + \x\x^\top\right)^{-1} \x\right)\left(\I + \x\x^\top\right)^{-1}  -
	 \left(\I + \x\x^\top\right)^{-1}  \x\x^\top \left(\I + \x\x^\top\right)^{-1} 
	 + \left(\I + \x\x^\top\right)^{-1}\right] \d\x
	\end{align*}
	}
	\item $f(\alpha) = \log |\K + \alpha \I|$, where $|\cdot|$ means the determinant. Derive $\frac{\partial f}{\partial \alpha}$\\
	{\blackblue \noindent
	We know that $\partial(\ln\det{X}) = {\rm tr}(X^{-1}\partial \X)$
	\begin{align*}
	\d \log |\K + \alpha \I | & =  {\rm tr}\left((\K + \alpha \I)^{-1}\d (\K + \alpha \I)\right)\\
	& = {\rm tr}\left((\K + \alpha \I)^{-1}(\d\alpha) \I\right)\\
	 & = {\rm tr}\left((\K + \alpha \I)^{-1}\right) \d\alpha\qquad \Longrightarrow \qquad \frac{\d  \log |\K + \alpha \I |}{\d \alpha} = {\rm tr}\left((K+\alpha\I)^{-1}\right)
	\end{align*}
	}
	
	%{\blackblue\noindent 
	%Using Jacobi's formula, we simply obtain 
	%\begin{align*}
	%\frac{\d  \log |\K + \alpha \I |}{\d \alpha} &=  {1\over |\K + \alpha \I |}\frac{\d |\K + \alpha \I |}{\d \alpha}\\
	%& = {1\over |\K + \alpha \I |}{\rm tr}\left({\rm adj}(\K + \alpha \I )\frac{\d (K+\alpha\I)}{\d\alpha}\right)\\
	%& = {1\over |\K + \alpha \I |}{\rm tr}\left({\rm adj}(\K + \alpha \I )\I\right)\\
	%& = {{\rm tr}\left({\rm adj}(\K + \alpha \I )\right)\over |\K + \alpha \I |}\\
	%& = {{\rm tr}\left({\rm adj}(\K + \alpha \I )\over |\K + \alpha \I |\right)}\\
	%& = {\rm tr}\left((K+\alpha\I)^{-1}\right).
	%\end{align*}
	%}
	\item $f(\bmu, \bSigma) = \log\big(\N(\a|\A\bmu, \S\bSigma\S^\top)\big)$, derive $\frac{\partial f}{\partial \bmu}$ and $\frac{\partial f}{\partial \bSigma}$,\\
	{\blackblue\noindent  
	%From The Matrix Cookboo, we use the two following equalities:
	%$$\frac{\d |\X|}{\d \X} = |\X|(\X^{-1})^T \quad\quad\&\quad\quad \frac{\d (\a^\top \X^{-1}\b)}{\d\X} = -(\X^{-1})^\top\a\b^\top(\X^{-1})^\top.$$
	%For simplicity, set $\m =  \A\mu$ and $\bOmega = \S\bSigma\S^\top$. Therefore, 
	%\begin{align*}
	%f(\m, \bOmega) = \frac{-1}{2}(\x-\m)^t\bOmega^{-1}(\x-\m) -{1\over 2}\log |\bOmega| +C.
	%\end{align*}
	%\begin{align*}
	%\frac{\d f}{\d \bmu} 
	%& = \frac{\d \m}{\d \bmu}\frac{\d f}{\d \m}  \\
	%& = \A\bOmega^{-1}(\x-\m) 
	%\end{align*}
	We remind that 
	$$f(\bmu, \bSigma) = \frac{-1}{2}(\a-\m)^t\bOmega^{-1}(\a-\m) -{1\over 2}\log |\bOmega| +C,$$
	where   $\m =  \A\mu$ and $\bOmega = \S\bSigma\S^\top$ and $C$ is a constant.
	\begin{align*}
	\d f
	& = \frac{-1}{2}\d(\a-\m)^\top\bOmega^{-1}(\a-\m) + \frac{-1}{2}(\a-\m)^\top\bOmega^{-1}\d(\a-\m) \\
	& = \frac{-1}{2}\d(\m)^\top\bOmega^{-1}(\a-\m) + \frac{1}{2}(\a-\m)^\top\bOmega^{-1}\d(\m)\\
	& = \frac{1}{2}(\a-\m)^\top(\bOmega^{-1})^\top\d(\m)+ \frac{1}{2}(\a-\m)^\top\bOmega^{-1}\d(\m) \quad\mbox{\red Since $\d(\m)^\top\bOmega^{-1}(\a-\m)$ is scaler.}\\
	& = (\a-\m)^\top\bOmega^{-1}\d(\m)\\
	& = (\a-\m)^\top\bOmega^{-1}\A\d\bmu \quad\Longrightarrow\quad {\red{\d f\over \d\bmu} = (\a-\m)^\top\bOmega^{-1}\A}
	\end{align*}
	%Therefore, $${\d f\over \d\bmu} = (\a-\m)^\top\bOmega^{-1}\A.$$
	\begin{align*}
	\d f
	& = -{1\over 2}(\a-\bmu)^\top (\d\bOmega^{-1})(\a-\bmu) -{1\over 2}\d(\log|\bOmega|) \\
	& =  -{1\over 2}(\a-\bmu)^\top \bOmega^{-1}(\d\bOmega)\bOmega^{-1}(\a-\bmu) -{1\over 2}{\rm tr}(\bOmega^{-1}(\d\bOmega))\\
	& =  -{1\over 2}(\a-\bmu)^\top \bOmega^{-1}\S(\d\bSigma)\S^\top\bOmega^{-1}(\a-\bmu) - {1\over 2}{\rm tr}(\bOmega^{-1}\S(\d\bSigma)\S^\top)\\
	& =  -{\rm tr}\left({1\over 2}\S^\top\bOmega^{-1}(\a-\bmu)(\a-\bmu)^\top \bOmega^{-1}\S(\d\bSigma)\right) - {1\over 2}{\rm tr}(\S^\top\bOmega^{-1}\S(\d\bSigma))
	\end{align*}
	Therefore, 
	\begin{align*}
	{\d f\over \d\bSigma} & = -{1\over 2}\S^\top\bOmega^{-1}(\a-\bmu)(\a-\bmu)^\top \bOmega^{-1}\S - {1\over 2}\S^\top\bOmega^{-1}\S\\
	& =-{1\over 2}\bSigma^{-1}\S^{-1}(\a-\bmu)(\a-\bmu)^\top (\S^{-1})^\top\bSigma^{-1} -{1\over 2}\bSigma^{-1}
	\end{align*}
	}
	\item $f(\bSigma) = \log\big(\N(\a|\b, \K\otimes \bSigma)\big)$ where $\otimes$ is the Kronecker product 
	(Hint: check Minka's notes).\\
	{\blackblue\noindent  
	Note that 
	$f(\bSigma) = -\frac{1}{2}(\a-\b)^t\bOmega^{-1}(\a-\b) -{1\over 2}\log |\bOmega| +C,$ where  $\bOmega = \K\otimes \bSigma$ and $C$ is a constant.
	\begin{align*}
	\d f
	& = -{1\over 2}(\a-\bmu)^\top (\d\bOmega^{-1})(\a-\bmu) -{1\over 2}\d\log|\bOmega| \\
	& =  -{1\over 2}(\a-\bmu)^\top \bOmega^{-1}(\d\bOmega)\bOmega^{-1}(\a-\bmu) -{1\over 2}{\rm tr}(\bOmega^{-1}(\d\bOmega))\\
	& =  -{1\over 2}{\rm tr}\left((\a-\bmu)^\top \bOmega^{-1}(\K\otimes \d\bSigma)\bOmega^{-1}(\a-\bmu)\right) - {1\over 2}{\rm tr}(\bOmega^{-1}(\K\otimes \d\bSigma)) \quad \mbox{\red Using $\d\bOmega = \K\otimes \d\bSigma$}\\
	& =  -{1\over 2}{\rm tr}\left(\bOmega^{-1}(\a-\bmu)(\a-\bmu)^\top \bOmega^{-1}(\K\otimes \d\bSigma)\right) - {1\over 2}{\rm tr}(\bOmega^{-1}(\K\otimes \d\bSigma))
	\end{align*}
	}
\end{enumerate}
\item~[2 points] Given the multivariate Gaussian probability density, $$p(\x|\bmu, \bSigma) = |2\pi \bSigma|^{-\frac{1}{2}}\exp\left(- (\x-\bmu)^\top\bSigma^{-1}(\x-\bmu)\right).$$ Show that the density function achieves the maximum when $\x = \bmu$.  \\
{\blackblue\noindent 
Since $\log(\cdot)$ is an increasing function, to make computation easy, we prove that $\log(p(\x|\bmu, \bSigma))$ takes it maximum in $\bmu$. 
\begin{align*}
{\d\over \d\x}\left(\log(p(\x|\bmu, \bSigma))\right)
& ={\d\over \d\x}\left( - (\x-\bmu)^\top\bSigma^{-1}(\x-\bmu) + C\right) \qquad\mbox{$C$ here is a constant}\\
& = -2\Sigma^{-1}(\x-\bmu) = 0\quad\quad \Longrightarrow x = \bmu
\end{align*}
Moreover,  since ${\d^2\over \d\x^2}\left(\log(p(\x|\bmu, \bSigma))\right) = -2\Sigma^{-1} \prec 0$, the function 
$\log(p(\x|\bmu, \bSigma))$ is a strictly concave function taking its maximum in $x = \bmu$, as desired. 
}
\item~[5 points] Show that $$\int \exp(-\frac{1}{2\sigma^2}x^2) \d x = \sqrt{2\pi \sigma^2}.$$ Note that this is about  how the normalization constant of the Gaussian density is obtained. Hint: consider its square and use double integral. \\
{\blackblue\noindent 
Set $I= \int \exp(-\frac{1}{2\sigma^2}x^2) \d x$. In the following we prove that $I^2 =  2\pi\sigma^2$.
\begin{align*}
I^2 & = (\int_{-\infty}^{\infty} \exp(-\frac{1}{2\sigma^2}x^2) \d x)^2 \\
& = \left(\int_{-\infty}^{\infty} \exp(-\frac{1}{2\sigma^2}x^2) \d x\right)\left( \int_{-\infty}^{\infty} \exp(-\frac{1}{2\sigma^2}y^2) \d y\right)\\
& = \int_{-\infty}^{\infty} \int_{-\infty}^{\infty} \exp\left(-\frac{1}{2\sigma^2}(x^2+y^2)\right)\d x\d y\\
& = \int_{0}^{2\pi} \int_{0}^{\infty} r\exp\left(-\frac{1}{2\sigma^2}r^2\right)\d r\d \theta\\
& = 2\pi\left[-\sigma^2\exp\left(-\frac{1}{2\sigma^2}r^2\right)\right]_0^\infty = 2\pi\sigma^2 \quad \Longrightarrow\quad I = \sqrt{2\pi \sigma^2}
\end{align*}
}
\item~[5 points] The gamma function is defined as $$\Gamma(x) = \int_0^\infty u^{x-1}e^{-u} \d u.$$ Show that $\Gamma(1) = 1$ and $\Gamma(x+1) = x\Gamma(x)$. Hint: using integral by parts. 
{\blackblue\noindent 
\begin{align*}
\Gamma(1) & =  \int_0^\infty e^{-u} \d u\\
& = \Big[-e^{-u}\Big]_{0}^{\infty} = 0 - (-1)  = 1.
\end{align*}
\begin{align*}
\Gamma(x+1) & =  \int_0^\infty u^x e^{-u} \d u\quad\quad \mbox{($U= u^x$ and $V' = e^{-u}$)}\\
& = \Big[-u^xe^{-u}\Big]_0^\infty + \int_0^\infty xu^{x-1} e^{-u} \d u\\
& = 0 + x\int_0^\infty u^{x-1} e^{-u} \d u = x\Gamma(x).
\end{align*}
}
\item~[2 points] By using Jensen's inequality with $f(x) = \log(x)$, show that for any collection of positive numbers $\{x_1, \ldots, x_N\}$,
$$\frac{1}{N}\sum_{n=1}^N x_n \ge \left(\prod_{n=1}^N x_n\right)^{\frac{1}{N}}.$$
{\blackblue\noindent 
Since $f(x) = -\log(x)$ is convex, using Jensen's inequality, we conclude 
%$$f\left(\frac{1}{N}\sum_{n=1}^N x_n\right) \leq \frac{1}{N}\sum_{n=1}^N f(x_n)$$
%or equivalently, 
\begin{align*}
-\log\left(\frac{1}{N}\sum_{n=1}^N x_n\right) 
& \leq \frac{1}{N}\sum_{n=1}^N -\log(x_n)\\
& = -\log\left(\left(\prod_{n=1}^N x_n\right)^{1/N}\right)
\end{align*}
Since $\log(\cdot)$ is an increasing monotonic function, it implies  
$$\frac{1}{N}\sum_{n=1}^N x_n \ge \left(\prod_{n=1}^N x_n\right)^{\frac{1}{N}}.$$
}
\item~[2 points] Given two probability density functions $p(\x)$ and $q(\x)$, show that $$\int p(\x)\log\frac{p(\x)}{q(\x)}\d \x \ge 0.$$
{\blackblue\noindent 
Since $f(x) = -\log(x)$ is convex, by Jensen's inequality, we have 
\begin{align*}
\int p(\x)\log\frac{p(\x)}{q(\x)}\d \x
& = \EE_{x\sim p}\left(-\log \frac{q(\x)}{p(\x)}\right)\\
\mbox{\red (using Jensen's inequality for $-\log(x)$)}\quad & \geq -\log\left(\EE_{x\sim p}\left(\frac{q(\x)}{p(\x)}\right)\right)\\
& = -\log\left(\int p(x)\left(\frac{q(\x)}{p(\x)}\right)\d x\right)\\
& = -\log\left(\int q(\x)\d x\right) = -\log 1 = 0.
\end{align*}
}
\item~[\textbf{Bonus}][5 points] Show that for any square matrix $\X \succ 0$, $\log\det\X$ is concave to $\X$. 
\end{enumerate}
{\blackblue\noindent Consider $\X$ as a vector in $\mathbb{R}^{2n}$ whose coordinates are indexed by ${ij}\in[n]\times [n]$. 
We know that $\frac{\d \log|\X|}{\d \X} = X^{-1}$ and consequently, 
$$\frac{\d^2 \log|\X|}{\d \X^2}  =\frac{\d \X^{-1}}{\d \X}.$$
By the Matrix Cookboo, 
$$\frac{\d (\X^{-1})_{kl}}{\d (\X)_{ij}} = -(X^{-1})_{ki}(X^{-1})_{jl}.$$
If we consider $\log(|\X|)$ as a function from $\mathbb{R}^{n^2}\longrightarrow \mathbb{R}$, then 
$L = \frac{\d^2 \log|\X|}{\d \X^2}$  is a $n^2\times n^2$ matrix whose rows and columns are indexed  by ${ij}\in[n]\times [n]$. 
In what follows, we prove that this matrix is semi-negative definite. Consider $\Z\in \mathbb{R}^{n^2}$, as a vector, such that its entries are indexed by indices  ${ij}\in[n]\times [n]$. To fulfill the proof, 
\begin{align*}
\Z^\top L\Z & = \sum_{kl}\sum_{ij}\Z_{kl}L_{kl,ij}\Z_{ij}\\
& = -\sum_{kl}\sum_{ij}\Z_{kl}(\X^{-1})_{ki}(\X^{-1})_{jl}\Z_{ij}\\
& = - \sum_{i}\sum_{l}\left(\sum_{j}(\X^{-1})_{jl}\Z_{ij}\right)\left(\sum_{k}\Z_{kl}(\X^{-1})_{ki}\right)\\
& = - \sum_{i}\sum_{l} (\Z\X^{-1})_{il}\left((\X^{-1})^\top\Z\right)_{il}\quad \mbox{\red Hereafter, we see $\X$ and $\Z$ as Matrices!}\\
& = - \sum_{i}\sum_{l} (\X^{-1}\Z)_{il}^2 \leq 0 \quad\quad \mbox{since $\X$ is symmetric.}
\end{align*}
}

\end{document}
%%% Local Variables:
%%% mode: latex
%%% TeX-master: t
%%% End:
